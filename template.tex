% !TEX TS-program = xelatex
% !TEX encoding = UTF-8 Unicode
%                                  ---by suchot
\documentclass[AutoFakeBold]{LZUthesis}



\begin{document}
%=====%
%
%封皮页填写内容
%
%=====%

% 标题样式 使用 \title{{}}; 使用时必须保证至少两个外侧括号
%  如: 短标题 \title{{第一行}},  
% 	      长标题 \title{{第一行}{第二行}}
%             超长标题\tiitle{{第一行}{...}{第N行}}

\title{{毕业论文}}



% 标题样式 使用 \entitle{{}}; 使用时必须保证至少两个外侧括号
%  如: 短标题 \entitle{{First row}},  
% 	      长标题 \entitle{{First row}{ Second row}}
%             超长标题\entitle{{First row}{...}{ Next N row}}
% 注意:  英文标题多行时 需要在开头加个空格 防止摘要标题处英语单词粘连。
\entitle{{Paper}}

\author{似然}
\major{数学}
\advisor{似然}
\college{萃英学院}
\grade{2016级}




\maketitle

%======%
%诚信说明页
%授权说明书
%======%
\makestatement
\frontmatter


%=====%
%论文(设计)成绩
%=====%

\supervisorcomment{导师评价你人很好}
\recommendedgrade{80}


\supervisorsignature{
	\raisebox{-10pt}{
		\includegraphics[width=60pt]{signature.pdf}
	}
}

\committeecomment{优秀}
\finalgrade{100}
\Grade



%英文摘要
\EnAbstract{Lanzhou University (LZU) is a key university in China. LZU was founded in 1909. It is one of the oldest universities in China. 
\par LZU has beautiful campuses, Library, Various laboratories. It has been actively involved in international academic exchange programs.It is the center of China Education and Research Network in northwest China region, through computer networks, LZU has faster and closer connection with the world.
}{LZU, key university,beautiful campus}
%中文摘要
\ZhAbstract{兰州大学是国家重点大学,成立于1909年,是中国有着悠久历史的大学之一。
	\par 兰州大学有美丽的校园、图书馆、各类实验室。兰州大学一直积极参与国际学术交流活动,是中国西北地区的教育和研究中心。通过计算机网络,兰州大学与世界的联系更加地快捷和紧密。
}{兰州大学,重点大学,美丽的校园}
%=======%
%插入引言
%=======%



%生成目录
\addtocontents{toc}{\protect\thispagestyle{empty}}
\tableofcontents
%插入图或表的目录

\renewcommand\listfigurename{插\ 图\ 目\ 录}
\renewcommand\listtablename{表\ 格\ 目\ 录}
\addtocontents{lof}{\protect\thispagestyle{empty}}
\listoffigures
%\thispagestyle{empty}
%\newpage
\addtocontents{lot}{\protect\thispagestyle{empty}}
\listoftables
%\thispagestyle{empty}
%\newpage
%文章主体
\mainmatter

\Intro
这里是引言部分


\chapter{研究背景}
\section{二级标题}
\subsection{三级标题}
\par \textbf{章节引用}示例,参见python代码见附录\ref{sec:code}
\par 听说\textbf{数学公式}输入好南

\begin{equation}
\begin{aligned}
\mathrm{KL}(p \| q) &=-\int p(\boldsymbol{x}) \ln q(\boldsymbol{x}) \mathrm{d} \boldsymbol{x}-\left(-\int p(\boldsymbol{x}) \ln p(\boldsymbol{x}) \mathrm{d} \boldsymbol{x}\right) \\
&=-\int p(\boldsymbol{x}) \ln \left\{\frac{q(\boldsymbol{x})}{p(\boldsymbol{x})}\right\} \mathrm{d} \boldsymbol{x}
\end{aligned}
\label{eq:KL}
\end{equation}

\par 引用KL散度公式,见公式\ref{eq:KL}。

\par 对于公式输入,我选择使用OCR识别,推荐一个小工具\href{https://accounts.mathpix.com/signup?referral_code=8ZTrrDwYv3}{Mathpix}(点击文档中的\href{https://accounts.mathpix.com/signup?referral_code=8ZTrrDwYv3}{Mathpix})

\par 听说\textbf{表格输入}也不太简单。学位论文通常使用三线表
\begin{table}[htbp] 
	\centering	
	\begin{tabular}{lcl} 
		\toprule 
		性别 & 身高 & 体重 \\ 
		\midrule 
		 女 & 165 & 60 \\ 
	     男 & 175 & 70 \\ 
		 女 & 161 & 55 \\ 
		\bottomrule 
	\end{tabular} 
\caption{\label{tab:test}示例表格} 
\end{table}
\par 听说你想\textbf{并排插入图像}
\begin{figure}[H]
	\begin{minipage}[t]{0.5\textwidth}
	\centering
	\tiny 图片1 \\
	\vspace{0.5cm}
	\includegraphics[scale=0.05]{anime.jpg}
\end{minipage}
\begin{minipage}[t]{0.5\textwidth}
	\centering 
	\tiny 图片2 \\
	\vspace{0.5cm}
	\includegraphics[scale=0.05]{anime.jpg}  
\end{minipage}	
\protect\caption{两张图并排 \label {fig:two-pics}}	
\end{figure}

\par 重要的事强调三遍,详细内容见ReadMe(求你,一定要看) \\
{\bfseries 编译方式:} XeLaTeX -->BibTeX --> XeLaTeX-->XeLaTeX \\
{\bfseries 编译方式:} XeLaTeX -->BibTeX --> XeLaTeX-->XeLaTeX \\
{\bfseries 编译方式:} XeLaTeX -->BibTeX --> XeLaTeX-->XeLaTeX
\chapter{研究方法}
\par 再多我也不会了,请学会正确使用搜索引擎。
\par 我适配的附录代码是python,如使用其他语言可前往cls文件修改代码展示部分$\rightarrow$language = Python。

\begin{lstlisting}[language = python]
def bin_search_recursively(l, first, last, n):
'''Binary search n in list l which has been sorted already, returns
the index if found, else returns None.'''
	if first > last:
		return None

	mid = (first + last) // 2 # Use / 2 if you're using Python 2
	if l[mid] > n:
		return bin_search_recursively(l, first, mid - 1, n)
	elif l[mid] < n:
		return bin_search_recursively(l, mid + 1, last, n)
	else:
		return mid


if __name__ == '__main__':
	sorted_num_list = list(range(1, 11))
	first = 0
	last = len(sorted_num_list) - 1
	n1 = 12
	n2 = 6

	print(bin_search_recursively(sorted_num_list, first, last, n1))
	print(bin_search_recursively(sorted_num_list, first, last, n2))

\end{lstlisting}

\chapter{研究结果}

\chapter{总结与讨论}

%论文后部
\backmatter


%=======%
%引入参考文献文件
%=======%
\bibdatabase{bib/database}%bib文件名称 仅修改bib/ 后部分
\printbib




\Appendix
\section{python代码}
\label{sec:code}


这里是附录页,附上你的程序或必要的相关知识\cite{partl2016}

{\bfseries 编译方式:} XeLaTeX -->BibTeX --> XeLaTeX-->XeLaTeX
\Thanks
这里是致谢页,你可以在这里致谢你的舍友,老师,朋友,或者我。


\end{document}